\documentclass[notes,11pt, aspectratio=169]{beamer}

\usepackage{pgfpages}
% These slides also contain speaker notes. You can print just the slides,
% just the notes, or both, depending on the setting below. Comment out the want
% you want.
\setbeameroption{hide notes} % Only slide
%\setbeameroption{show only notes} % Only notes
%\setbeameroption{show notes on second screen=right} % Both

\usepackage{helvet}
% \usepackage[default]{lato}
\usepackage{array}


\usepackage{tikz}
\usepackage{verbatim}
\setbeamertemplate{note page}{\pagecolor{yellow!5}\insertnote}
\usetikzlibrary{positioning}
\usetikzlibrary{snakes}
\usetikzlibrary{calc}
\usetikzlibrary{arrows}
\usetikzlibrary{decorations.markings}
\usetikzlibrary{shapes.misc}
\usetikzlibrary{matrix,shapes,arrows,fit,tikzmark}
\usepackage{amsmath}
\usepackage{mathpazo}
\usepackage{hyperref}
\usepackage{lipsum}
\usepackage{multimedia}
\usepackage{graphicx}
\usepackage{multirow}
\usepackage{graphicx}
\usepackage{dcolumn}
\usepackage{bbm}
\newcolumntype{d}[0]{D{.}{.}{5}}

\usepackage{changepage}
\usepackage{appendixnumberbeamer}
\newcommand{\beginbackup}{
   \newcounter{framenumbervorappendix}
   \setcounter{framenumbervorappendix}{\value{framenumber}}
   \setbeamertemplate{footline}
   {
     \leavevmode%
     \hline
     box{%
       \begin{beamercolorbox}[wd=\paperwidth,ht=2.25ex,dp=1ex,right]{footlinecolor}%
%         \insertframenumber  \hspace*{2ex} 
       \end{beamercolorbox}}%
     \vskip0pt%
   }
 }
\newcommand{\backupend}{
   \addtocounter{framenumbervorappendix}{-\value{framenumber}}
   \addtocounter{framenumber}{\value{framenumbervorappendix}} 
}


\usepackage{graphicx}
\usepackage[space]{grffile}
\usepackage{booktabs}

% These are my colors -- there are many like them, but these ones are mine.
\definecolor{blue}{RGB}{0,114,178}
\definecolor{red}{RGB}{213,94,0}
\definecolor{yellow}{RGB}{240,228,66}
\definecolor{green}{RGB}{0,158,115}

% % Enviroments
% \newtheorem{defin}{Definition.}
% \newtheorem{teo}{Theorem. }
% \newtheorem{lema}{Lemma. }
% \newtheorem{coro}{Corolary. }
% \newtheorem{prop}{Proposition. }
% \theoremstyle{definition}
% \newtheorem{examp}{Example. }
% % \numberwithin{problem}{subsection} 

\hypersetup{
  colorlinks=false,
  linkbordercolor = {white},
  linkcolor = {blue}
}


%% I use a beige off white for my background
\definecolor{MyBackground}{RGB}{255,253,218}

%% Uncomment this if you want to change the background color to something else
%\setbeamercolor{background canvas}{bg=MyBackground}

%% Change the bg color to adjust your transition slide background color!
\newenvironment{transitionframe}{
  \setbeamercolor{background canvas}{bg=yellow}
  \begin{frame}}{
    \end{frame}
}

\setbeamercolor{frametitle}{fg=blue}
\setbeamercolor{title}{fg=black}
\setbeamertemplate{footline}[frame number]
\setbeamertemplate{navigation symbols}{} 
\setbeamertemplate{itemize items}{-}
\setbeamercolor{itemize item}{fg=blue}
\setbeamercolor{itemize subitem}{fg=blue}
\setbeamercolor{enumerate item}{fg=blue}
\setbeamercolor{enumerate subitem}{fg=blue}
\setbeamercolor{button}{bg=MyBackground,fg=blue,}
\setbeamercolor{theotem}{fg=blue} 

% If you like road maps, rather than having clutter at the top, have a roadmap show up at the end of each section 
% (and after your introduction)
% Uncomment this is if you want the roadmap!
\AtBeginSubsection[]
{
   \begin{frame}
       \frametitle{Roadmap of Talk}
       \tableofcontents[currentsection]
   \end{frame}
}


\setbeamercolor{section in toc}{fg=blue}
\setbeamercolor{subsection in toc}{fg=red}
\setbeamersize{text margin left=1em,text margin right=1em} 

\newenvironment{wideitemize}{\itemize\addtolength{\itemsep}{10pt}}{\enditemize}

\usepackage{environ}
\NewEnviron{videoframe}[1]{
  \begin{frame}
    \vspace{-8pt}
    \begin{columns}[onlytextwidth, T] % align columns
      \begin{column}{.58\textwidth}
        \begin{minipage}[t][\textheight][t]
          {\dimexpr\textwidth}
          \vspace{8pt}
          \hspace{4pt} {\Large \sc \textcolor{blue}{#1}}
          \vspace{8pt}
          
          \BODY
        \end{minipage}
      \end{column}%
      \hfill%
      \begin{column}{.42\textwidth}
        \colorbox{green!20}{\begin{minipage}[t][1.2\textheight][t]
            {\dimexpr\textwidth}
            Face goes here
          \end{minipage}}
      \end{column}%
    \end{columns}
  \end{frame}
}

\title[]{\textcolor{blue}{ECON 736 Presentation \\ Assortative Matching with Large Firms}}
\author[MVB]{}
\institute[UW-Madison]{Mitchell Valdes-Bobes}

\date{\today}


\begin{document}
%%% TIKZ STUFF
\tikzset{   
        every picture/.style={remember picture,baseline},
        every node/.style={anchor=base,align=center,outer sep=1.5pt},
        every path/.style={thick},
        }
\newcommand\marktopleft[1]{%
    \tikz[overlay,remember picture] 
        \node (marker-#1-a) at (-.3em,.3em) {};%
}
\newcommand\markbottomright[2]{%
    \tikz[overlay,remember picture] 
        \node (marker-#1-b) at (0em,0em) {};%
}
\tikzstyle{every picture}+=[remember picture] 
\tikzstyle{mybox} =[draw=black, very thick, rectangle, inner sep=10pt, inner ysep=20pt]
\tikzstyle{fancytitle} =[draw=black,fill=red, text=white]
%%%% END TIKZ STUFF

% Title Slide
\begin{frame}
	\maketitle
\end{frame}
% Outline Slide
\begin{frame}
	\frametitle{Roadmap of Talk}
	\tableofcontents
\end{frame}

% INTRO
\section{Introduction}
% Indroduction Slide
\begin{frame}{Introduction}
\end{frame}

% Motivation Slide
\begin{frame}{Motivation}
\end{frame}

% Unnamed Slide with research questions
\begin{frame}{Research Questions}
	\begin{wideitemize}
	\item \textcolor{red}{\textbf{Research Question}}
	\begin{wideitemize}
		\item Provide a unifying theory of production with a trade-off between hiring more vs better workers.
	\end{wideitemize}

	\pause

	\item \textcolor{red}{\textbf{Results}}
	\begin{wideitemize}
		\item Sorting condition that captures the trade-off between quantity and quality of workers.
		\item Characterization of matching in equilibrium.
		\item When is matching assortative \textbf{(PAM)} or \textbf{(NAM)}?
		\item Under what conditions more productive firms hire more workers in equilibrium?
	\end{wideitemize}
\end{wideitemize}

\end{frame}

% MODEL
\section{Model}

\subsection{Model set-up}
% Model setup silde 1 Demographics
\begin{frame}{Model Setup: Demographics}

	\begin{wideitemize}
		\item  \textbf{Workers} indexed by \textit{unidimensional} skill $x\in[\underline{x},  \bar{x} ]\subset \mathbb{R}_+$
		\begin{itemize}
			\item  \textit{CDF} $H^w(x)$ and \textit{PDF} $h^w$  
		\end{itemize}
		\item  \textbf{Firms} indexed by \textit{unidimensional} productivity $y\in[\underline{y},  \bar{y} ]\subset \mathbb{R}_+$
    \begin{itemize}
        \item \textit{CDF} $H^f(x)$ and \textit{PDF} $h^f$  
    \end{itemize}
\end{wideitemize}
\end{frame}

% Model setup silde 2 Preferences
\begin{frame}{Model Setup: Preferences}

	
	\begin{wideitemize}
		\item Linear utility model.
		\item \textbf{Workers} care about their wage and there is no disutility of work.
		\item  \textbf{Firms} maximize their profits.
	\end{wideitemize}

\end{frame}

% Model setup slide 3 production function
\begin{frame}{Model Setup:  Production Function}
	\begin{wideitemize}
		\item The output produced by a firm of type \textcolor{red}{$y$} that hires \textcolor{green}{$l$} workers of type \textcolor{red}{$x$} is: \[F(\textcolor{red}{x,y},\textcolor{green}{l,r})\]
		\begin{itemize}
			\item \textcolor{green}{$r$} the fraction of \textcolor{red}{$y$}'s resources dedicated to \textcolor{red}{$x$} type workers.
			\item \textcolor{red}{$(x,y)$} are quality variables and \textcolor{green}{$(l,r)$} are quantity variables.
		% 	\item The resource might reflect the time endowment of an entrepreneur who spends time interacting with and supervising her employees.
		% 	\begin{itemize}
		% 		\item Under this interpretation buying resources in the market (like capital) is excluded from the problem.
		% 	\end{itemize}
		% 	\item The quality might refer to the value of the final output.
		\end{itemize}
		\pause
		\item $F$ is strictly increasing and strictly concave in $(l,r)$, $0$ resources produce $0$ output, and standard Inada conditions apply.
		%  \item  Consider the cross-partials of the output function:
		% \begin{itemize}
		%   \item $F_{xy}$ If positive, means that higher firm types have, ceteris paribus, a higher marginal return for matching with higher worker types.
		%   \item $F_{yl}$ If it is large, it means that higher firm types have a higher marginal valuation.
		%   \item $F_{xr}$ Expresses how the marginal product of resources spent on workers varies with worker type.
		%   \item  $F_{lr}$ Captures the extent to which additional labor decreases the value of output.
		% \end{itemize}
		\item  $F$ has constant returns to scale in $l$ and $r$.
		\pause
		\item  We can write $F$ in terms of \textbf{intensity} $\theta=l/r$ :\[f(x,y,\theta):=F(x,y,\theta,1)\quad \implies \quad F(x,y,l,r) = rf(x,y,\theta)\]
		% \item Either $F$ or $f$ can be used as the primitive of the model.
	\end{wideitemize}
\end{frame}


\subsection{Equilibrium}
% Equilibrium slide 1: Firm's Problem
\begin{frame}{Equilibrium}
	\begin{wideitemize}
		\item The equilibrium concept used is the competitive equilibrium.
		\pause
		\item \textcolor{red}{Firm's problem:}
		\begin{wideitemize}
			% \item  Firms choose two distributions
			\item  Distribution of workers hired by $y$ $\mathcal{L}^y(x) = \int_{\underline{x}}^xl^y(\tilde{x})dH^w(\tilde{x})$ 
			\item  Distribution of firm $y$ resources $\mathcal{R}^y(x) = \int_{\underline{x}}^xr^y(\tilde{x})dH^w(\tilde{x})$
			\pause
			\item  For any $x\in[\underline{x}, \bar{x}]$ $l^y(x) = \theta^y(x)r^y(x)$ which means 
			\begin{equation}\label{labor_demand}
			\mathcal{L}^y(x) = \int_{\underline{x}}^x\theta^y(\tilde{x})d\mathcal{R}^y(\tilde{x}).
			\end{equation}
			\item The total output of the firm can be writen as:
			\[\int_{\underline{x}}^{\overline{x}} f\left(x, y, \theta^{y}(x)\right) d \mathcal{R}^{y}(x) = \int_{\underline{x}}^{\overline{x}} F\left(x, y, l^y(x), r^y(x)\right) d \mathcal{H}^{w}(x)\]
			\item  Firms maximize the difference between output produced and wages paid to workers.  
		\end{wideitemize}
	\end{wideitemize}
\end{frame}


% Slide feasible demand
\begin{frame}{Equilibrium}
	\begin{wideitemize}
		\item \textcolor{red}{Feasible Labor Demand}
		\begin{wideitemize}
			\item Consider an interval of worker types $(x',x]$
			\item The demand of firm $y$ for those workers is $\mathcal{L}^y(x) - \mathcal{L}^y(x')$
			\pause
			\item This implies a way to evaluate if a labor demand schedule $\{\mathcal{L}^y\}_{y\in \mathcal{Y}}$ is feasible:\[\int_{y}\left[\mathcal{L}^{y}(x)-\mathcal{L}^{y}\left(x^{\prime}\right)\right] d H^{f} \leq H^{w}(x)-H^{w}\left(x^{\prime}\right) \qquad \forall  (x',x]\subseteq\mathcal{X}\]
		\end{wideitemize}
	\end{wideitemize}
	% Note that for any interval of worker types $(x', x]$, a firm of type $y$ has a demand for such workers of $\mathcal{L}^y(x)-\mathcal{L}^y(x')$ then the aggregate demand for such workers is the integral over all firms $y\in \mathcal{Y}$. 
	% This implies a way to evaluate if a labor demand schedule $\{\mathcal{L}^y\}_{y\in \mathcal{Y}}$ is feasible:$$\int_{y}\left[\mathcal{L}^{y}(x)-\mathcal{L}^{y}\left(x^{\prime}\right)\right] d H^{f} \leq H^{w}(x)-H^{w}\left(x^{\prime}\right) \qquad \forall  (x',x]\subseteq\mathcal{X}$$
\end{frame}

% Slide equilibrium definition
\begin{frame}{Equilibrium Definition}
	\begin{wideitemize}
		\item An equilibrium is a tuple of functions $\left(w, \theta^{y}, \mathcal{R}^{y}, \mathcal{L}^{y}\right)$ consisting of a non-negative wage schedule $w(x)$ as well as intensity functions $\theta^{y}(x)$ and resource allocations $\mathcal{R}^{y}(x)$ with associated feasible labor demands $\mathcal{L}^{y}(x)$ (determined as in \eqref{labor_demand}) such that:
		\pause
		\begin{wideitemize}
			\item \textbf{Optimality:} Given the wage schedule $w(x)$, for any $y$, the combination $\left(\theta^{y}, \mathcal{R}^{y}\right)$ solves:$$\max _{\theta^y, \mathcal{R}^{y}} \int\left[f\left(x, y, \theta^{y}(x)\right)-w(x) \theta^{y}(x)\right] d \mathcal{R}^{y}(x)$$
			\pause
			\item \textbf{Market Clearing:} For any $(x',x]\subseteq\mathcal{X}$
			\[\text{If } w(x)>0 \text{ a.e in } (x', x]  \quad \implies \quad   \int_{y}\left[\mathcal{L}^{y}(x)-\mathcal{L}^{y}\left(x^{\prime}\right)\right] d H^{f} = H^{w}(x)-H^{w}\left(x^{\prime}\right) \]
		\end{wideitemize}
	\end{wideitemize}
\end{frame}

\subsubsection{Sorting Condition}
% Slide equilibrium characterization
\begin{frame}{Equilibrium Characterization}
	\begin{wideitemize}
		\item  When do better firms hire \textbf{better} workers?
		\item  How are wages determined?
		\item  When do better firms employ \textbf{more} workers?
		\item  How is that affected by technological change?
	\end{wideitemize}
	
\end{frame}

% Slide Equilibrium Assortativity
\begin{frame}{Equilibrium Assortativity}
	\textcolor{green}{\textbf{Definition}} (Assortative Matching) 
	\begin{wideitemize}
	\item We say that matching between firms and workers is PAM (NAM) if higher type firms hire higher type workers, i.e., $y>y'$ then, $x$ in the support of $\mathcal{L}^y$ and $x'$ in the support of $\mathcal{L}^{y'}$ only if $x\geq (\leq) x'$.
	\end{wideitemize}
\end{frame}

% Slide Sketch of proof Proposition 1
\begin{frame}[label=prop_1_slide]{Equilibrium Characterization}
	\only<1>{
		\textcolor{red}{\textbf{Proposition 1}} \label{prop1}
		\begin{wideitemize}
			\item If output $F$ is strictly increasing in $x$ and $y$ and the type distributions have nonzero continuous densities, then almost all active firm types y hire exactly one worker type and reach unique size $l(y)$ in an \textbf{assortative} equilibrium.
			\item There is an injective matching function $\mu:\mathcal{\tilde{X}}\to\mathcal{\tilde{Y}}$, between the subset of hired workers and active firms.
		\end{wideitemize}
	}
	\only<2>{
		The proof have two parts:
		\begin{wideitemize}
			\item First we show that for every hired worker the combination $(x,\theta^y(x))$ solves \hyperlink{appendix_steps_prop_1_1<1>}{\beamergotobutton{Details}}
			\begin{equation}\label{simple_max}
			(x,\theta^y(x)) \in \arg\max \left\{ f(\tilde{x}, y, \tilde{\theta})-\tilde{\theta} w(\tilde{x}) \right \} \qquad \forall x\in \text{supp}\mathcal{R}^y
			\end{equation}
			\begin{wideitemize}
				\item An implication is that in equilibirum if a worker is hired then all wokers that are more productive must have strictly possitive wages.
			\end{wideitemize}
		\end{wideitemize}
		}
		\only<3>{
		\begin{wideitemize}
			\item Second, assume that a firm hires two different workers $x'<x$, if the equilibirum is \textbf{PAM} then that firm must be the only firm that hire workers in $[x',x]$.
			\item If there is only one firm active in $[x',x]$ then the aggregate labor demand has zero measure and by market clearing $w(\hat{x})=0$ for all $\hat{x} \in (x',x)$.
			\item This contradicts what we showed.
		\end{wideitemize}
	}
\end{frame}

% Slide Conditions for assortative equilibrium
\begin{frame}[label=condition_assort]{Conditions for Assortative Equilibrium}
	\begin{wideitemize}
		\only<1>{
		\item We can restrict our attention to the problem \[max_{x,\theta(x)}\:  f(x, \mu(x), \theta(x))-\theta(x) w(x)\] 
		
		\item Taking first and second order conditios \hyperlink{appendix_algebra_assort_char_1<1>}{\beamergotobutton{Details}} we arrive at the expression:
			\[\textcolor{red}{\mu^{\prime}(x)}\left[f_{\theta \theta}f_{x y} - f_{y \theta}\left(f_{x \theta}-\frac{f_{x}}{\theta(x)}\right)\right] < 0 \]
		}
		\only<2>{
		\item Note that a \textbf{PAM} equilibrium requires $\mu'(x)>0$, this implies a necessary condition: \[f_{\theta \theta}f_{x y} - f_{y \theta}\left(f_{x \theta}-\frac{f_{x}}{\theta(x)}\right) < 0\]

		\item We can write this condition in terms of $F$ \hyperlink{appendix_algebra_from_f_to_F<1>}{\beamergotobutton{Details}} to deal with the potential endogeneity of $\theta(x)$: \[F_{x y} > \frac{F_{y l} F_{x r}}{F_{l r}}\]
		\pause
		\item We have found a necessary condition for the equilibirum matching to be \textbf{PAM}, tunrs out that this is also a sufficient condition.
		}
	\end{wideitemize}
\end{frame}

% Slide Main results: Proposition 2
\begin{frame}[label=main_result]{Main Asssortativity Result}
	\only<1>{
		\textcolor{red}{\textbf{Proposition 2}} \label{prop1}
		\begin{wideitemize}
			\item A necessary and sufficient condition to have equilibria with positive assortative matching is that the following inequality holds:

			\begin{equation}\label{pam_cond}
				F_{x y} > \frac{F_{y l} F_{x r}}{F_{l r}}    
			\end{equation}
			
			for all $(x, y, l, r) \in \mathbb{R}_{++}^{4} .$
			
			The opposite inequality provides a necessary and sufficient condition for negative assortative matching.
		\end{wideitemize}
	}
	\only<2>{
		
		\begin{wideitemize}
			\item Since the firm's problem is quasi-linear then Pareto optimality requires output maximization. 
			\item This is the key idea behind the proof: if \eqref{pam_cond} holds then the output of any not positive assortative allocation can be strictly improved implying that it is not an equilibrium.
		\end{wideitemize}
		}
		\only<3>{
		\begin{wideitemize}
			\item Consider some matching $(x,y,\theta)$ such that a total measure $r$ of resources is deployed in this match, the output generated is \[F(x,y,\theta r, r) = r f(x,y,\theta)\]
			\item We can show \hyperlink{appendix_proof_prop_2<1>}{\beamergotobutton{Details}} that the marginal change of shifting an optimal measure of workers of type $x$  from firm $y$ to firm $\hat{y}$:

			\begin{equation}\label{beta_defin}
			\beta(\hat{y};x,y,\theta) = f(x,\hat{y}, \hat{\theta}) - \hat{\theta}f_\theta(x,y,\theta) \quad \text{where} \quad f_\theta(x,y,\theta)=f_\theta(x,\hat{y},\hat{\theta})
			\end{equation}

		\end{wideitemize}
	}
	\only<4>{
		\begin{wideitemize}
			\item Suppose that equilibrium matching is not \textbf{PAM}, i.e  $x_{1}$ is matched to $y_{1}$ at intensity $\theta_{1}$ and $x_{2}$ to $y_{2}$ at intensity $\theta_{2}$, but $x_{1}>x_{2}$ while $y_{1}<y_{2}$, for this match to be efficient the following two inequalities mus \textbf{simultaneously} hold:

			\begin{equation}\label{beta_y1}
				\beta(y_1; x_2, y_2, \theta_2) \leq \beta(y_1; x_1, y_1, \theta_1) 
			\end{equation}
			\begin{equation}\label{beta_y2}
				\beta(y_2; x_1, y_1, \theta_1) \leq \beta(y_2; x_2, y_2, \theta_2) 
			\end{equation}
		\item To finalize the proof we show that \eqref{beta_y1}, \eqref{beta_y2} and \eqref{pam_cond} leads to a contradiction \hyperlink{appendix_proof_prop_2<2>}{\beamergotobutton{Details}}.
		\end{wideitemize}
	}
\end{frame}

\subsubsection{Equilibrium Assignment}
% Slide Equilibrium Assignment
\begin{frame}[label=eq_assignment]{Equilibrium Assignment}
	\only<1>{
		\begin{wideitemize}
			\item This model deals with both the intensive and the extensive margin. 
			\item Assortativity is not enough to characterize who matches with whom in equilibrium.
			\item Firms could hire more or fewer workers. 
			\item We will characterize the matching with a system of differential equations.
		\end{wideitemize}
	}
	\only<2>{
			\begin{wideitemize}
				\item Since the matching $y = \mu(x)$, is singled valued we have  $l^y(x) = \theta(x)\mathbbm{1}_{\{y=\mu(x)\}}$
				\item We can show \hyperlink{appendix_eq_assignment<1>}{\beamergotobutton{Details}} that in equilibrium the demand of labor in any interval $(x, \overline{x}]$ is 
				\[\int_{y}\left[\mathcal{L}^{y}(x)-\mathcal{L}^{y}\left(x^{\prime}\right)\right] d H^{f} = \int_{\mu(x)}^{\overline{y}}{ \theta(\mu^{-1}(y)) dH^f }_
				= \textcolor{green}{\underbrace{ H^w(\overline{x}) -  H^w(x) }_{\text{by Labor Market Clearing}}} \]
				\item Differentiating w.r.t $x$ both sides and solving for $\mu'(x)$:
				$$\mu'(x) = \frac{\mathcal{H}(x)}{\theta(x)}\qquad \text{with} \qquad \mathcal{H}(x)= \frac{h^{w}(x)}{h^{f}(\mu(x))}$$    
			\end{wideitemize}	
	}
	\only<3>{
			\begin{wideitemize}
				\item From the first order condition of the problem we have:\[w^{\prime}(x)=\frac{f_{x}}{\theta(x)} \]
				\item Using the differenciated version of the FOC and some algebra \hyperlink{appendix_eq_assignment<2>}{\beamergotobutton{Details}} we get:$$
				\theta^{\prime}(x)=\frac{\mathcal{H}(x) f_{x y}}{\theta(x) f_{x \theta}}
				$$
		\end{wideitemize}
	}
	\only<4>{
		\begin{wideitemize}
			\item The system of differential equations:
				$$
				\left\{\begin{aligned}
				\mu^{\prime}(x) &=\frac{\mathcal{H}(x)}{\theta(x)} \\
				w^{\prime}(x) &=\frac{f_{x}}{\theta(x)} \\
				\theta^{\prime}(x) &=\frac{\mathcal{H}(x) f_{x y}}{\theta(x) f_{x \theta}}
				\end{aligned}\right.
				$$
				characterizes the equilibrium.
		\end{wideitemize}
	}

\end{frame}

% SIMULATION
\section{Simulation}
\subsection{Simulation Strategy}
\begin{frame}{Simulation Strategy}
	\begin{wideitemize}
		\only<1>{
		\item We want to numerically solve a system of ODE's.
		\item We need an initial condition:
			$$\mu(\underline{x}) = \underline{\mu} \qquad \text{and} \qquad \theta(\underline{x}) = \underline{\theta}$$
		\item Positive assortative matching gives us one initial condition:
		
			$$\mu(\underline{x}) = \underline{y}$$
		
		\item But we are still unable to pindown $\underline{\theta}$.
		}
		\only<2>{
		\item We know a terminal condition for $\mu(x)$:
		
			\[\mu(\overline{x}) = \overline{y}\]
		
		\item This turns the initial value problem into a boundary condition problem. 
		
		\item We can solve using a shooting algorithm.
		
		\item The idea of the shooting algorithm is to select an initial value for $\underline{\theta}$, solve the system, and compare the obtained value of $\mu(\overline{x})$ with $\overline{y}$ and iterativelly update $\underline{\theta}$ until convergence.		
		}
	\end{wideitemize}
\end{frame}

\subsection{Simulation Results}
\begin{frame}{Simulation Results}
	\begin{wideitemize}
		
		\item{} To simulate the model we will use the following production function:
				\[f(x,y,\theta) = \left(\omega_{A} x^{\left(1-\sigma_{A}\right) / \sigma_{A}}+\left(1-\omega_{A}\right) y^{\left(1-\sigma_{A}\right) / \sigma_{A}}\right)^{\sigma_{A} /\left(1-\sigma_{A}\right)} \theta^{\omega_{B}}\]
		\pause
		\item{}	Computing the sorting condition for this production function we get:
				\begin{equation}
					-\frac{\textcolor{red}{\left(1-\sigma _A\right)} \left(1-\omega _A\right) \omega
					_A x^{\frac{1}{\sigma _A}} y^{\frac{1}{\sigma _A}} \theta
				^{\omega _B} \left(\omega _A x^{\frac{1}{\sigma
				_A}-1}+\left(1-\omega_A\right) y^{\frac{1}{\sigma
				_A}-1}\right){}^{\frac{\sigma _A}{1-\sigma _A}}}{\sigma _A
				\left(\omega _A \left(y x^{\frac{1}{\sigma _A}}-x
				y^{\frac{1}{\sigma _A}}\right)+x y^{\frac{1}{\sigma
				_A}}\right){}^2} > 0
				\end{equation}
			\pause
		\item{} Clearly the condition for \textbf{PAM} holds if $\sigma_A < 1$ and we will have \textbf{NAM} if $\sigma_A > 1$.
	\end{wideitemize}
\end{frame}

\begin{frame}[label=change_omega_A]{Effect of changing $\omega_A$} 
	\only<1>{
	\begin{wideitemize}
		\item{} When $\omega_A = 0.5$ workers and firms are equally weighted.
		\item{} Fully symmetric model, mathing $\mu(x) = x$, reach constant size
	\end{wideitemize}
	}
	\only<2>{
	\begin{wideitemize}
		\item{} $\omega_A \in (0.5, 1]$ worker type is more determinant in production.
		\item{} The size effect dominates the type effect $\implies$ matching is concave and firm size is increasing.
	\end{wideitemize}
	}
	\only<3>{
	\begin{wideitemize}
		\item{} $\omega_A \in [0, 0.5)$ firm type is more determinant in production.
		\item{} The type effect dominates the size effect $\implies$ matching is convex and firm size is decreasing.
	\end{wideitemize}
	}	
	\centering
	\only<1>{\resizebox{12.5cm}{!}{
		\includegraphics{figures/plot_positive_ω_A_1.pdf}
	}
	}
	\only<2>{\resizebox{12.5cm}{!}{
		\includegraphics{figures/plot_positive_ω_A_2.pdf}
	}
	}
	\only<3>{\resizebox{12.5cm}{!}{
		\includegraphics{figures/plot_positive_ω_A_3.pdf}
	}
	}
	
	\begin{wideitemize}
		\item{} \textbf{Parametrization} $x,y \sim U[0,1]$, $\omega_B = 0.5$ and $\sigma_A = 0.9$
		\only<3>{\item{} \hyperlink{appendix_wage_change_omega_A}{\beamergotobutton{Effect in wages}}}
	\end{wideitemize}
\end{frame}

\begin{frame}[label=change_sigma_A]{Effect of changing $\sigma_A$} 
	\begin{wideitemize}
		\item{} Higher values of $\sigma_A$ means that higher type workers are more attractive.
		\item{}  Since the supply of labor constrained $\implies$ stealing of workers.
	\end{wideitemize}
	\centering
	\only<1>{\resizebox{12.5cm}{!}{
		\includegraphics{figures/plot_positive_σ_A_1.pdf}
	}
	}
	\only<2>{\resizebox{12.5cm}{!}{
		\includegraphics{figures/plot_positive_σ_A_2.pdf}
	}
	}
	\only<3>{\resizebox{12.5cm}{!}{
		\includegraphics{figures/plot_positive_σ_A_3.pdf}
	}
	}
	
	\begin{wideitemize}
		\item{} \textbf{Parametrization} $x,y \sim U[0,1]$, $\omega_B = 0.5$ and $\omega_A = 0.75$
		\only<3>{\item{} \hyperlink{appendix_wage_change_sigma_A}{\beamergotobutton{Effect in wages}}}
	\end{wideitemize}
\end{frame}

\section{Appendix}
\begin{frame}[label=appendix_steps_prop_1_1]{}
	\begin{itemize}
		\only<1>{
			\item Suppose that a firm $y$ that uses strategy $(\theta^y, \mathcal{R}^y)$ to solve the problem 
				\begin{equation}\label{complex_max}
				\max _{\theta y, \mathcal{R}^{y}} \int\left[f\left(x, y, \theta^{y}(x)\right)-w(x) \theta^{y}(x)\right] d \mathcal{R}^{y}(x)
				\end{equation}
			\item Proceed by contradiction, and suppose that there is a set of hired workers $\tilde{\mathcal{X}}$ for which their assigned resources do not solve
				\[(x,\theta^y(x)) \in \arg\max \left\{ f(\tilde{x}, y, \tilde{\theta})-\tilde{\theta} w(\tilde{x}) \right \} \qquad \forall x\in \text{supp}\mathcal{R}^y\]
			\item Define:
				\[\mathcal{X}^*=\left\{x\in \mathcal{X}\mid (x,\theta^*(x)) \in \arg\max \left\{ f(\tilde{x}, y, \tilde{\theta})-\tilde{\theta} w(\tilde{x}) , \text{for some } \theta^*\right \} \right\}$$ $$\tilde{\mathcal{X}} = \mathcal{X} / \mathcal{X}^*\]
		}
		\only<2>{
			\item Consider any $x^* \in \mathcal{X}^*$ and a strategy where the firm places or the resources on $x^*$ at intensity $\theta^*$ we have:
				\[f(x,y,\theta^y(x)) = f(x^*, y, \theta^*) \qquad \forall x\in \mathcal{X}^*\]
				\[f(x,y,\theta^y(x)) < f(x^*, y, \theta^*) \qquad \forall x\in \tilde{\mathcal{X}}\]
			\item Note that the profits pf the firm are:
			\begin{align*}
				\int_{\mathcal{X}^*}\left[f(x,y,\theta^y(x)) - w(x)\theta^y(x)\right]d\mathcal{R}^y(x) &+ \int_{\tilde{\mathcal{X}}}\left[f(x,y,\theta^y(x)) - w(x)\theta^y(x)\right]d\mathcal{R}^y(x)\\
				&<\\
				\int_{\mathcal{X}^*}\left[f(x^*,y,\theta^*) - w(x^*)\theta^*\right]d\mathcal{R}^y(x) &+ \int_{\tilde{\mathcal{X}}}\left[f(x^*,y,\theta^*) - w(x^*)\theta^*\right]d\mathcal{R}^y(x)
			\end{align*}

	\item The firm can strictly increase its profits, therefore the original strategy is not a solution of \eqref{complex_max}.  \hyperlink{prop_1_slide<2>}{\beamergotobutton{Back}}
	}
\end{itemize}
\end{frame}

\begin{frame}[label=appendix_algebra_assort_char_1]{slide 1}
	\hyperlink{condition_assort<1>}{\beamergotobutton{Back}}
\end{frame}

\begin{frame}[label=appendix_algebra_from_f_to_F]{slide 2}
	\hyperlink{condition_assort<2>}{\beamergotobutton{Back}}
\end{frame}

\begin{frame}[label=appendix_proof_prop_2]{slide 3}
	\only<1>{
		\hyperlink{main_result<3>}{\beamergotobutton{Back}}
	}
	\only<2>{
		\hyperlink{main_result<4>}{\beamergotobutton{Back}}
	}
\end{frame}

\begin{frame}[label=appendix_eq_assignment]{slide 4}
	\only<1>{
		\hyperlink{eq_assignment<2>}{\beamergotobutton{Back}}
	}
	\only<2>{
		\hyperlink{eq_assignment<3>}{\beamergotobutton{Back}}
	}
\end{frame}

\begin{frame}[label=appendix_wage_change_omega_A]{Use it to intimidate audiences!}
	\hyperlink{change_omega_A<3>}{\beamergotobutton{Back}}
\end{frame}

\begin{frame}[label=appendix_wage_change_sigma_A]{Use it to intimidate audiences!}
	\hyperlink{change_sigma_A<3>}{\beamergotobutton{Back}}
\end{frame}


\end{document}