\documentclass[notes,11pt, aspectratio=169]{beamer}

\usepackage{pgfpages}
% These slides also contain speaker notes. You can print just the slides,
% just the notes, or both, depending on the setting below. Comment out the want
% you want.
\setbeameroption{hide notes} % Only slide
%\setbeameroption{show only notes} % Only notes
%\setbeameroption{show notes on second screen=right} % Both

\usepackage{helvet}
% \usepackage[default]{lato}
\usepackage{array}


\usepackage{tikz}
\usepackage{verbatim}
\setbeamertemplate{note page}{\pagecolor{yellow!5}\insertnote}
\usetikzlibrary{positioning}
\usetikzlibrary{snakes}
\usetikzlibrary{calc}
\usetikzlibrary{arrows}
\usetikzlibrary{decorations.markings}
\usetikzlibrary{shapes.misc}
\usetikzlibrary{matrix,shapes,arrows,fit,tikzmark}
\usepackage{amsmath}
\usepackage{mathpazo}
\usepackage{hyperref}
\usepackage{lipsum}
\usepackage{multimedia}
\usepackage{graphicx}
\usepackage{multirow}
\usepackage{graphicx}
\usepackage{dcolumn}
% \usepackage{bbm}
\newcolumntype{d}[0]{D{.}{.}{5}}

\usepackage{changepage}
\usepackage{appendixnumberbeamer}
\newcommand{\beginbackup}{
   \newcounter{framenumbervorappendix}
   \setcounter{framenumbervorappendix}{\value{framenumber}}
   \setbeamertemplate{footline}
   {
     \leavevmode%
     \hline
     box{%
       \begin{beamercolorbox}[wd=\paperwidth,ht=2.25ex,dp=1ex,right]{footlinecolor}%
%         \insertframenumber  \hspace*{2ex} 
       \end{beamercolorbox}}%
     \vskip0pt%
   }
 }
\newcommand{\backupend}{
   \addtocounter{framenumbervorappendix}{-\value{framenumber}}
   \addtocounter{framenumber}{\value{framenumbervorappendix}} 
}


\usepackage{graphicx}
\usepackage[space]{grffile}
\usepackage{booktabs}

% These are my colors -- there are many like them, but these ones are mine.
\definecolor{blue}{RGB}{0,114,178}
\definecolor{red}{RGB}{213,94,0}
\definecolor{yellow}{RGB}{240,228,66}
\definecolor{green}{RGB}{0,158,115}

\hypersetup{
  colorlinks=false,
  linkbordercolor = {white},
  linkcolor = {blue}
}


%% I use a beige off white for my background
\definecolor{MyBackground}{RGB}{255,253,218}

%% Uncomment this if you want to change the background color to something else
%\setbeamercolor{background canvas}{bg=MyBackground}

%% Change the bg color to adjust your transition slide background color!
\newenvironment{transitionframe}{
  \setbeamercolor{background canvas}{bg=yellow}
  \begin{frame}}{
    \end{frame}
}

\setbeamercolor{frametitle}{fg=blue}
\setbeamercolor{title}{fg=black}
\setbeamertemplate{footline}[frame number]
\setbeamertemplate{navigation symbols}{} 
\setbeamertemplate{itemize items}{-}
\setbeamercolor{itemize item}{fg=blue}
\setbeamercolor{itemize subitem}{fg=blue}
\setbeamercolor{enumerate item}{fg=blue}
\setbeamercolor{enumerate subitem}{fg=blue}
\setbeamercolor{button}{bg=MyBackground,fg=blue,}



% If you like road maps, rather than having clutter at the top, have a roadmap show up at the end of each section 
% (and after your introduction)
% Uncomment this is if you want the roadmap!
\AtBeginSection[]
{
   \begin{frame}
       \frametitle{Roadmap of Talk}
       \tableofcontents[currentsection]
   \end{frame}
}


\setbeamercolor{section in toc}{fg=blue}
\setbeamercolor{subsection in toc}{fg=red}
\setbeamersize{text margin left=1em,text margin right=1em} 

\newenvironment{wideitemize}{\itemize\addtolength{\itemsep}{10pt}}{\enditemize}

\usepackage{environ}
\NewEnviron{videoframe}[1]{
  \begin{frame}
    \vspace{-8pt}
    \begin{columns}[onlytextwidth, T] % align columns
      \begin{column}{.58\textwidth}
        \begin{minipage}[t][\textheight][t]
          {\dimexpr\textwidth}
          \vspace{8pt}
          \hspace{4pt} {\Large \sc \textcolor{blue}{#1}}
          \vspace{8pt}
          
          \BODY
        \end{minipage}
      \end{column}%
      \hfill%
      \begin{column}{.42\textwidth}
        \colorbox{green!20}{\begin{minipage}[t][1.2\textheight][t]
            {\dimexpr\textwidth}
            Face goes here
          \end{minipage}}
      \end{column}%
    \end{columns}
  \end{frame}
}

\title[]{\textcolor{blue}{ECON 736 Presentation \\ Assortative Matching with Large Firms}}
\author[MVB]{}
\institute[UW-Madison]{Mitchell Valdes-Bobes}

\date{\today}


\begin{document}
%%% TIKZ STUFF
\tikzset{   
        every picture/.style={remember picture,baseline},
        every node/.style={anchor=base,align=center,outer sep=1.5pt},
        every path/.style={thick},
        }
\newcommand\marktopleft[1]{%
    \tikz[overlay,remember picture] 
        \node (marker-#1-a) at (-.3em,.3em) {};%
}
\newcommand\markbottomright[2]{%
    \tikz[overlay,remember picture] 
        \node (marker-#1-b) at (0em,0em) {};%
}
\tikzstyle{every picture}+=[remember picture] 
\tikzstyle{mybox} =[draw=black, very thick, rectangle, inner sep=10pt, inner ysep=20pt]
\tikzstyle{fancytitle} =[draw=black,fill=red, text=white]
%%%% END TIKZ STUFF

% Title Slide
\begin{frame}
	\maketitle
\end{frame}


% INTRO
\section{Introduction}

% MODEL
\section{Model}
\subsection{Model set-up}
\subsection{Equilibrium}
\subsubsection{Characterization of Equilibrium}
\subsubsection{Assortativity Characterization}
\subsubsection{Equilibrium Assignment}

% SIMULATION
\section{Simulation}
\subsection{Simulation Strategy}
\begin{frame}{Title}
	\begin{wideitemize}
		
		\item{} To simulate the model we will use the following production function:

			\begin{equation}\label{prod_funct}
				f(x,y,\theta) = \left(\omega_{A} x^{\left(1-\sigma_{A}\right) / \sigma_{A}}+\left(1-\omega_{A}\right) y^{\left(1-\sigma_{A}\right) / \sigma_{A}}\right)^{\sigma_{A} /\left(1-\sigma_{A}\right)} \theta^{\omega_{B}}
			\end{equation}
	
		\item{}	Computing condition \ref{pam_cond_simple} for this production function we get:
				\begin{equation}
					-\frac{\textcolor{red}{\left(1-\sigma _A\right)} \left(1-\omega _A\right) \omega
					_A x^{\frac{1}{\sigma _A}} y^{\frac{1}{\sigma _A}} \theta
				^{\omega _B} \left(\omega _A x^{\frac{1}{\sigma
				_A}-1}+\left(1-\omega_A\right) y^{\frac{1}{\sigma
				_A}-1}\right){}^{\frac{\sigma _A}{1-\sigma _A}}}{\sigma _A
				\left(\omega _A \left(y x^{\frac{1}{\sigma _A}}-x
				y^{\frac{1}{\sigma _A}}\right)+x y^{\frac{1}{\sigma
				_A}}\right){}^2} > 0
				\end{equation}
	
		\item{} Clearly the condition for \textbf{PAM} holds if $\sigma_A < 1$ and we will have \textbf{NAM} if $\sigma_A > 1$.
		\end{wideitemize}
\end{frame}

\subsection{Simulation Results}
\begin{frame}{Effect of changing $\omega_A$} 
	\only<1>{
	\begin{wideitemize}
		\item{} When $\omega_A = 0.5$ workers and firms are equally weighted.
		\item{} Fully symmetric model, mathing $\mu(x) = x$, reach constant size
	\end{wideitemize}
	}
	\only<2>{
	\begin{wideitemize}
		\item{} $\omega_A \in (0.5, 1]$ worker type is more determinant in production.
		\item{} The size effect dominates the type effect $\implies$ matching is concave and firm size is increasing.
	\end{wideitemize}
	}
	\only<3>{
	\begin{wideitemize}
		\item{} $\omega_A \in [0, 0.5)$ firm type is more determinant in production.
		\item{} The type effect dominates the size effect $\implies$ matching is convex and firm size is decreasing.
	\end{wideitemize}
	}	
	\centering
	\only<1>{\resizebox{12.5cm}{!}{
		\includegraphics{figures/plot_positive_ω_A_1.pdf}
	}
	}
	\only<2>{\resizebox{12.5cm}{!}{
		\includegraphics{figures/plot_positive_ω_A_2.pdf}
	}
	}
	\only<3>{\resizebox{12.5cm}{!}{
		\includegraphics{figures/plot_positive_ω_A_3.pdf}
	}
	}
	
	\begin{wideitemize}
		\item{} \textbf{Parametrization} $x,y \sim U[0,1]$, $\omega_B = 0.5$ and $\sigma_A = 0.9$
		\only<3>{\item{} \hyperlink{appendix_wage_change_omega_A}{\beamergotobutton{Effect in wages}}}
	\end{wideitemize}
\end{frame}

\section{Appendix}
\begin{frame}[label=appendix_wage_change_omega_A]{Use it to intimidate audiences!}
	\begin{wideitemize}
	  \item[] Now you can make it clear you've done a shitload of work
		\begin{itemize}
		\item[]  without having to show everything! \hyperlink{appendix_start}{\beamergotobutton{Back}}
		\end{itemize}
	  \item[] You label a frame with the \texttt{[label=name]} option, and then point a link to it
	  \item[] You can make an object a link using the \texttt{\textbackslash hyperlink\{label\}\{object\}} command
	\end{wideitemize}
  \end{frame}


\end{document}